\documentclass[french]{article}
\usepackage[T1]{fontenc}
\usepackage[utf8]{inputenc}
\usepackage{lipsum}
\usepackage{lmodern}
\usepackage{geometry}
\usepackage{babel}
\usepackage{graphicx}
\usepackage{lastpage}
\usepackage{ragged2e}
\usepackage{enumitem}
\usepackage[normalem]{ulem}

\geometry{
 	a4paper,
 	total={210mm,297mm},
 	left=20mm,
 	right=20mm,
 	top=20mm,
 	bottom=20mm,
}

\usepackage{fancyhdr}
\pagestyle{fancy}
\setlist[enumerate,1]{leftmargin=2cm}

\lhead{Lucien Badoux, Andrea Colza, Adrian Di Pietro, \\Kevin Moreira, François Quellec}
\chead{}
\rhead{PRO: Cahier des charges}
\renewcommand{\headrulewidth}{0.4pt}
\renewcommand{\footrulewidth}{0.4pt}

\begin{document}
	\centering
	\LARGE{\textbf{Cahier des charges}}
  \large

	\justify

  \section{Objectifs}
  L'objectif premier de notre projet est de pouvoir appliqué différents algorithmes sur des graphes construits graphiquement par l'utilisateur.
  Ainsi l'utilisateur pourra visualiser les différentes étapes de l'algorithme choisi de manière à mieux le comprendre.
  En plus du côté éducatif qui s'applique sur des graphes à petit nombre de noeuds (il serait laborieux de construire des graphes trop grand avec une interface graphique),
   nous allons inclure la possibilité d'appliquer ces algorithmes sur des graphes directement importés depuis des fichiers CSV.


  \section{Public cible}
  Cette application se place avant tout dans le domaine éducatif, elle concernerait surtout les professeurs et les étudiants qui appliquent les différents algorithmes décrits plus bas.
  Cela permettrait à un élève d'obtenir la correction de différents problèmes de graphes et réseaux afin de mieux les comprendre étape par étape.
  Dans un second temps cette application peut convenir à des professionels afin d'appliquer sur de gros graphes ces algorithmes et récupérer le résultat dans fichier qu'ils pourront
  ensuite traiter/analyser facilement depuis un outil comme Excel.

	\section{Spécifications}
  Afin d'implémenter notre application nous allons utilisé le langage de programmation JAVA. Les interfaces graphiques seront implémentées à l'aide de Java FX.



	\section{Fonctionalités}
    Cette section décrit les fonctionalités attendues du programme :
		\begin{enumerate}
			\item Mode Graphique (usage scolaire)
			\begin{enumerate}
        \item Click \& Drop avec verrous:
        \begin{enumerate}
          \item Sommets (Nom)
          \item Arêtes (pondéré ou non)
          \item Arcs (pondéré ou non)
          \item Ajout d'une zone de texte
          \item (Optionnelle) Arcs avec capacité
        \end{enumerate}
				\item Exportation du graphe au formats (png, jpg)
        \item Resolution des différents algorithmes étapes par étapes (avec rendu visuel)
        \item (Future) Représentation de graphes importé au format CSV
        \item (Future) Replacement automatique des sommets afin d'obtenir un graphe planaire le plus possible.
			 \end{enumerate}

		   \item Mode non Graphique (usage professionnel)
       \begin{enumerate}
 				\item Importation de graphes au format CSV.
 				\item Exportation des graphes après manipulation au format CSV.
        \item Affichage textuelle des manipulations
 				\item (Optionnelle) Manipulation des graphes (ajouter/supprimer des noeuds/arêtes)
 			\end{enumerate}

      \item Fonctions Utilitaires
			\begin{enumerate}
				\item Sauvegarde et ouverture de graphes (à spécifier)
				\item Bouton d'aide
				\item (Optionnelle) Gestion d'onglets différents
			\end{enumerate}
		\end{enumerate}

		\subsection{Application des Algorithmes}
			Les algorithmes suivant seront implémentés et pourront être exécutés depuis
      l'application :
			\begin{itemize}
				\item Parcours en profondeur
        \item Parcours en largeur
        \item Algorithme de Kosaraju
        \item Algorithme de Tarjan
        \item Algorithme de Prim
        \item Algorithme de Kruskal
        \item Algorithme de Chu-Liu
        \item Algorithme de Bellman-Ford
        \item Algorithme de Dijkstra
        \item Algorithme de Johnson
        \item Tri topologique
        \item (Optionnelle) Calcul des chemins critiques
        \item (Optionnelle) Algorithme de Floyd-Warshall
        \item (Optionnelle) Algorithme de Ford-Fulkerson
        \item (Optionnelle) Algorithme de Busacker-Gowen
				\item (Optionnelle) Problème de transbordement
			\end{itemize}

  \section{Charte graphique}
    \subsection{Interface Edition}
    \begin{minipage}{\linewidth}
    \center
       \includegraphics[scale=0.5]{Interface_Edition.jpg}
    \end{minipage}
    \\
    \\
    Cette vue est l'interface principale de notre programme, c'est d'ici que nous pourrons dessiner des graphes et accéder aux différentes fonctionnalités utilitaires(créer, sauvegarder, exporter, importer, aide, etc).
    Le panel d'outil à gauche permet de selectionner les composants que l'on souhaite dessiner (arêtes, sommets), en haut on trouve les boutons des fonctionnalités utilitaires citées plus haut.
    Juste en dessous de ce panel apparaissent les différents onglets et enfin le bouton en haut à gauche permet de passer en mode Algorithme.

    \subsection{Interface Algorithme}
    \begin{minipage}{\linewidth}
    \center
       \includegraphics[scale=0.5]{Interface_Algorithme.jpg}
    \end{minipage}
    \\
    \\
    Une fois que l'utilisateur aura créé son graphe il pourra appliquer une panoplie d'algorithmes sur celui-ci à partir de l'interface suivante.
    Il lui faudra alors choisir l'algorithme  à appliquer à partir du panel sur la gauche.
    Ensuite les boutons situés en haut de l'interface permettront d'appliquer l'algorithme pas-à-pas. Les états des structures de données(à gauche) et la description(à droite) de chaque étape seront affichés dans les deux Panels du bas.
    Enfin une fois les algorithmes appliqués, il sera possible de basculer le graphe résultant en mode édition à partir du bouton en haut à droite.

    \subsection{Interfaces CSV}
    \begin{minipage}{\linewidth}
    \center
       \includegraphics[scale=0.8]{Interface_CSV.jpg}
    \end{minipage}
    \\
    \\
    A partir de l'interface édition il sera possible d'importer un graphe en CSV. La fênetre suivante permet ensuite de pouvoir sélectionner un type de représentation.
    On pourra ensuite spécifier le fichier à importer.
    Le programme charge alors le dit fichier et nous donne quelque information sur celui-ci afin de vérifier sa conformité.
    Il est ensuite possible d'appliquer un algorithme dessus.
    Enfin une fois l'algorithme appliqué la fenêtre suivante nous permettra de sélectionner le chemin dans lequel on souhaite enregistrer le graphe résultant.

  \section{Planification}
  Nous avons planifié les différentes étapes de notre projet avec l'outil GanttProject, un pdf récapitulatif est joint à ce dossier.
  La répartition des heures est résumée ci-dessous :
  \newline
  \begin{minipage}{\linewidth}
     \includegraphics[scale=0.25]{repartition_des_heures.jpg}
  \end{minipage}



\end{document}
