\documentclass[french]{article}
\usepackage[T1]{fontenc}
\usepackage[utf8]{inputenc}
\usepackage{lipsum}
\usepackage{lmodern}
\usepackage{geometry}
\usepackage{babel}
\usepackage{graphicx}
\usepackage{lastpage}
\usepackage{ragged2e}
\usepackage{enumitem}
\usepackage[normalem]{ulem}

\geometry{
 	a4paper,
 	total={210mm,297mm},
 	left=20mm,
 	right=20mm,
 	top=20mm,
 	bottom=20mm,
}

\usepackage{fancyhdr}
\pagestyle{fancy}
\setlist[enumerate,1]{leftmargin=2cm}

\lhead{Lucien Badoux, Andrea Colza, Adrian Di Pietro, \\Kevin Moreira, François Quellec}
\chead{}
\rhead{PRO: Cahier des charges}
\renewcommand{\headrulewidth}{0.4pt}
\renewcommand{\footrulewidth}{0.4pt}

\begin{document}
	\centering
	\LARGE{\textbf{Cahier des charges}}
  \large

	\justify

  \section{Objectifs}
  L'objectif premier de notre projet est de pouvoir appliqué différents algorithmes sur des graphes construits graphiquement par l'utilisateur.
  Ainsi l'utilisateur pourra visualiser les différentes étapes de l'algorithme choisi de manière à mieux le comprendre.
  En plus du côté éducatif qui s'applique sur des graphes à petit nombre de noeuds (il serait laborieux de construire des graphes trop grand avec une interface graphique),
   nous allons inclure la possibilité d'appliquer ces algorithmes sur des graphes directement importés depuis des fichiers CSV.


  \section{Public cible}
  Cette application se place avant tout dans le domaine éducatif, elle concernerait surtout les professeurs et les étudiants qui appliquent les différents algorithmes décrits plus bas.
  Cela permettrait à un élève d'obtenir la correction de différents problèmes de graphes et réseaux afin de mieux les comprendre étape par étape.
  Dans un second temps cette application peut convenir à des professionels afin d'appliquer sur de gros graphes ces algorithmes et récupérer le résultat dans fichier qu'ils pourront
  ensuite traiter/analyser facilement depuis un outil comme Excel. 

	\section{Spécifications}

  \section{Charte graphique}


	\section{Fonctionalités}
    Cette section décrit les fonctionalités attendues du programme :
		\begin{enumerate}
			\item Mode Graphique (usage scolaire)
			\begin{enumerate}
        \item Click \& Drop avec verrous:
        \begin{enumerate}
          \item Sommets (Nom)
          \item Arêtes (pondéré ou non)
          \item Arcs (pondéré ou non)
          \item Ajout d'une zone de texte
          \item (Optionnel) Arcs avec capacité
        \end{enumerate}
				\item Exportation du graphe au formats (png, jpg)
        \item Resolution des différents algorithmes étapes par étapes (avec rendu visuel)
        \item (Future) Représentation de graphes importé au format CSV
			 \end{enumerate}

		   \item Mode non Graphique (usage professionnel)
       \begin{enumerate}
 				\item Importation de graphes au format CSV.
 				\item Exportation des graphes après manipulation au format CSV.
        \item Affichage textuelle des manipulations
 				\item (Optionnelle) Manipulation des graphes (ajouter/supprimer des noeuds/arêtes)
 			\end{enumerate}

      \item Fonctions Utilitaires
			\begin{enumerate}
				\item Sauvegarde et ouverture de graphes (à spécifier)
				\item Bouton d'aide
				\item (Optionnelle) Gestion d'onglets différents
			\end{enumerate}
		\end{enumerate}

		\subsection{Application des Algorithmes}
			Les algorithmes suivant seront implémentés et pourront être exécuté depuis
      l'application :
			\begin{itemize}
				\item Parcours en profondeur
        \item Parcours en largeur
        \item Algorithme de Kosaraju
        \item Algorithme de Tarjan
        \item Algorithme de Prim
        \item Algorithme de Kruskal
        \item Algorithme de Chu-Liu
        \item Algorithme de Bellman-Ford
        \item Algorithme de Dijkstra
        \item Algorithme de Johnson
        \item Tri topologique
        \item (Optionnel) Algorithme de Floyd-Warshall
        \item (Optionnel) Algorithme de Ford-Fulkerson
        \item (Optionnel) Algorithme de Busacker-Gowen
				\item (Optionnel) Problème de transbordement
			\end{itemize}

	\section{Planification}



\end{document}
